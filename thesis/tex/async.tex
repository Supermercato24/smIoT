\subsection{Sincrono}
\label{sub:asyncSincrono}

Le architetture dei grandi sistemi informativi sono, ancora oggi, nella quasi totalità dei casi, costruite su piattaforme \textit{client}/\textit{server}.
In questo paradigma due host interagiscono tra loro: il primo richiede dati e il secondo li fornisce.
Questo tipo di comunicazione è denominato \verb+sincrono+ e segue questo flusso:
\begin{enumerate}
  \item il \textit{client} invia una richiesta
  \item il \textit{server} elabora la richiesta e invia la risposta
\end{enumerate}
Durante tutto il periodo di interazione, gli interlocutori risultano impegnati esclusivamente nella transazione e quindi impossibilitati ad eseguire altre operazioni.
Se il \textit{server} dovesse risultare impegnato nel rispondere, l'esperienza per il \textit{client} sarebbe imprevista.
Le entità che possono interagire in un dato istante sono solo due, il che rende più difficoltosa la realizzazione di applicazioni che richiedono comunicazioni
1:N o N:N.
Nonostante i limiti, questo modello ha riscosso un incredibile successo.
Ciò è da imputare sia alla semplicità del modello stesso sia alla sua adattabilità a moltissimi ambiti applicativi.

Un'applicazione di consumo delle risorse basata su questo modello, interroga costantemente un servizio per aggiornare i propri valori.
Perché un'applicazione del genere risulti utile, deve effettuare una serie di interrogazioni con un'alta frequenza per garantire un aggiornamento dei valori con un ritardo massimo stabilito.
Minore è il ritardo massimo accettabile, maggiore è la frequenza delle interrogazioni e di conseguenza lo spreco di risorse (se la risorsa non è cambiata) \cite{PubSub}.

\subsection{Asincrono}
\label{sub:asyncAsincrono}

Un paradigma di comunicazione dove esiste un totale disaccoppiamento tra produttore e consumatore dell'informazione è denominato \verb+asincrono+.
I consumatori (\textit{subscriber}) possono esprimere il loro interesse verso un sottoinsieme di risorse ed essere successivamente notificati dell'aggiornamento generato da un produttore (\textit{publisher}).
Tale comunicazione asincrona viene identificata come \textit{publish}/\textit{subscribe}.
L'introduzione di questo nuovo modello all'interno di applicazioni esistenti ha permesso di migliorare l'uso delle risorse da parte delle applicazioni stesse, lasciando inalterato il modo di operare degli utenti.
