\lettrine{S}{incronizzare} notevoli quantità di dati in tempo reale è un aspetto critico per molte piattaforme online.
L'utilizzo di una connessione persistente implica l'uso continuo della connessione dati, \textit{parsing} dello stream input e \textit{encoding} dello stream output.
Compiti non indifferenti a seguito della stima tra i 50 e 100 miliardi dei dispositivi connessi entro il 2020.
Fenomeni di \textit{spinlock} o di \textit{polling} per ottenere un risorsa aggiornata, dovrebbero essere abbandonati a favore di una comunicazione asincrona tra gli interlocutori.

Il lavoro di tesi si è sviluppato con un'introduzione ai paradigmi di comunicazione e background dei protocolli nel capitolo \ref{ch:background}.
Nel capitolo \ref{ch:solution} sono state proposte due soluzioni implementative diverse con una panoramica tecnica dei software attuali presso l'azienda Supermercato24.
Queste applicazioni sono state poi valutate tramite test nel capitolo \ref{ch:evaluation}.

Le prove effettuate hanno confermato come sia estremamente complesso costruire un modello che caratterizzi al meglio il risparmio di risorse nella creazione e nel mantenimento della connessione.
Tra queste variabili, di cui bisogna tenere necessariamente conto, c'è la reale possibilità che i pacchetti dati possano non arrivare a destinazione.
Di conseguenza la qualità del servizio deve essere una prerogativa irrinunciabile.

\verb+HTTP+, ampiamente utilizzato nella rete, nonostante sia avido di risorse garantisce un buon punto di partenza.
L'introduzione del \verb+WebSocket+ che implementa la comunicazione asincrona, ha permesso a questo protocollo di mantenersi ancora \textit{standard de facto} in Internet.
La nuova versione di questo protocollo aprirà le porte a nuovi scenari, come già sta facendo Google per la modifica dei documenti in tempo reale in \textit{Drive}.

\verb+MQTT+, di contro, è stato progettato per soddisfare da subito queste esigenze offrendo una serie di metodi nativi sgravando lo sviluppatore da questo compito.
La garanzia della consegna del messaggio prevista su tre livelli a secondo dalla criticità e la leggerezza dello stream dati, lo pone come un'alternativa più che valida.

Dalle verifiche sperimentali sono emerse le differenze tra i due protocolli sulle risorse utilizzate.
In un'ora di test il consumo di batteria con \verb+HTTP+ è il triplo rispetto a \verb+MQTT+.
Considerato il tempo trascorso dalla CPU eseguendo il codice nello \textit{user space}, denota la leggerezza del secondo protocollo.

Grazie a questa analisi è stato approvata la sperimentazione del protocollo \verb+MQTT+ come vettore nella sincronizzazione dati lato e-commerce.
Un altro problema attuale in Supermercato24 è il consumo significativo del gestionale dei fattorini.
La comunicazione della posizione geografica tramite chiamate \verb+AJAX+, esaurisce velocemente la batteria dei dispositivi; problema comune ad altri servizi simili come Deliveroo, Foodora e Just Eat.
L'utilizzo di \textit{power bank} a supporto diventa quindi indispensabile per poter affrontare più ordini.
Concluso con successo il test del progetto \ref{sec:smiot} verrà valutata in futuro questa migrazione.
Uno sviluppo interessante sarebbe confrontare questa possibile evoluzione ed i benefici ottenuti.
Un'altra opportunità potrebbe essere l'approfondimento di questi i test con l'attivazione della compressione e della crittografia, sacrificando la potenza computazionale a favore della connessione dati.

Concludendo, i protocolli proposti rappresentano soltanto un piccolo insieme.
L'affidabilità e la compatibilità restano gli ambiti in cui la ricerca dovrà muoversi in preparazione della crescita esponenziale nell'ambiente \verb+IoT+.
