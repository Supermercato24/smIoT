Tramite software \textit{Wireshark} si è analizzato i principali pacchetti scambiati, mostrando prima il pacchetto \verb+HTTP+ e poi \verb+MQTT+.

\bigskip
\noindent
Si sono usate le seguenti specifiche:
\begin{itemize}
\item analisi dettagliata dei pacchetti tramite software \textit{Wireshark}
\item connessioni senza compressione e senza cifratura
\item \textit{back-end} installato localmente emulando i dati di produzione
\end{itemize}

\subsection{Creazione della connessione}
\label{sub:packetCreazione}

Tramite protocollo \verb+HTTP+ viene effettuato l'\textit{handshake} necessario per creare il \textit{WebSocket}.
Tra l'indirizzo \textit{0x0040} e \textit{0x0070} si può notare il cambio di protocollo, come evidenziato nella sezione \ref{subsec:httpWebsocket}.
\verb+MQTT+ fornisce tutti i parametri necessari, si noti il token di sessione per autenticare la richiesta nel byte all'indirizzo \textit{0x0060}.

\begin{lstlisting}[language=bash, label={lst:packetCreazioneHttp}, captionpos=b, caption={HTTP Upgrade Request per creare il socket WS}, basicstyle=\scriptsize\ttfamily]
0000   02 00 00 00 45 00 00 e3 3e f2 40 00 40 06 00 00  ....E...>.@.@...
0010   7f 00 00 01 7f 00 00 01 0b b8 dc 7f 01 d3 94 ac  ................
0020   8d de 93 69 80 18 31 c3 fe d7 00 00 01 01 08 0a  ...i..1.........
0030   33 3c a5 2b 33 3c a5 2a 48 54 54 50 2f 31 2e 31  3<.+3<.*HTTP/1.1
0040   20 31 30 31 20 53 77 69 74 63 68 69 6e 67 20 50   101 Switching P
0050   72 6f 74 6f 63 6f 6c 73 0d 0a 55 70 67 72 61 64  rotocols..Upgrad
0060   65 3a 20 77 65 62 73 6f 63 6b 65 74 0d 0a 43 6f  e: websocket..Co
0070   6e 6e 65 63 74 69 6f 6e 3a 20 55 70 67 72 61 64  nnection: Upgrad
0080   65 0d 0a 53 65 63 2d 57 65 62 53 6f 63 6b 65 74  e..Sec-WebSocket
0090   2d 41 63 63 65 70 74 3a 20 66 63 72 6f 6f 2b 47  -Accept: fcroo+G
00a0   33 50 30 62 71 32 52 66 53 38 45 5a 39 6f 37 35  3P0bq2RfS8EZ9o75
00b0   72 6d 4b 45 3d 0d 0a 53 65 63 2d 57 65 62 53 6f  rmKE=..Sec-WebSo
00c0   63 6b 65 74 2d 45 78 74 65 6e 73 69 6f 6e 73 3a  cket-Extensions:
00d0   20 70 65 72 6d 65 73 73 61 67 65 2d 64 65 66 6c   permessage-defl
00e0   61 74 65 0d 0a 0d 0a                             ate....
\end{lstlisting}

\begin{lstlisting}[language=bash, label={lst:packetCreazioneMqtt}, captionpos=b, caption={MQTT Creazione socket}, basicstyle=\scriptsize\ttfamily]
0000   02 00 00 00 45 00 00 6c fe c2 40 00 40 06 00 00  ....E..l..@.@...
0010   7f 00 00 01 7f 00 00 01 de 5b 07 5b 56 1f e3 58  .........[.[V..X
0020   b5 ad d1 73 80 18 31 d7 fe 60 00 00 01 01 08 0a  ...s..1..`......
0030   33 65 87 64 33 65 87 62 10 36 00 04 4d 51 54 54  3e.d3e.b.6..MQTT
0040   04 c2 00 14 00 20 30 39 31 65 31 30 38 38 31 36  ..... 091e108816
0050   33 66 34 64 65 61 39 30 35 66 64 32 32 32 61 64  3f4dea905fd222ad
0060   34 61 66 36 61 66 00 06 66 6f 6f 62 61 72 00 00  4af6af..foobar..
\end{lstlisting}

\subsection{Sottoscrizione al topic}
\label{sub:packetSottoscrizione}

La sottoscrizione al topic \verb+HTTP+ viene effettuata in \verb+Javascript+ e viene spezzata in due richieste \verb+TCP+.
\verb+MQTT+ permette la sottoscrizione simultanea a più argomenti contemporaneamente con una singola richiesta.
Questo pacchetto viene inviato immediatamente dopo la creazione della connessione.
Il dettaglio dei pacchetti sono apprezzabili nell'appendice \ref{app:packetSottoscrizione}.

\subsection{Mantenimento della connessione}
\label{sub:packetMantenimento}

Il probe \textit{KeepAlive} in \verb+HTTP+ serve solo per controllare che la connessione sia ancora attiva, mentre \verb+MQTT+ si aspetta anche una risposta da parte il \textit{server}.

\begin{lstlisting}[language=bash, label={lst:packetMantenimentoHttq}, captionpos=b, caption={HTTP Probe KeepAlive}, basicstyle=\scriptsize\ttfamily]
0000   02 00 00 00 45 00 00 37 13 b4 40 00 40 06 00 00  ....E..7..@.@...
0010   7f 00 00 01 7f 00 00 01 0b b8 dc 7f 01 d3 98 5d  ...............]
0020   8d de 93 a7 80 18 31 c1 fe 2b 00 00 01 01 08 0a  ......1..+......
0030   33 3d 06 83 33 3d 06 83 81 01 33                 3=..3=....3
\end{lstlisting}

\begin{lstlisting}[language=bash, label={lst:packetMantenimentoMqtt}, captionpos=b, caption={MQTT Probe KeepAlive}, basicstyle=\scriptsize\ttfamily]
0000   02 00 00 00 45 00 00 36 26 3c 40 00 40 06 00 00  ....E..6&<@.@...
0010   7f 00 00 01 7f 00 00 01 de 5b 07 5b 56 1f e3 9c  .........[.[V...
0020   b5 ad d1 7c 80 18 31 d7 fe 2a 00 00 01 01 08 0a  ...|..1..*......
0030   33 65 e0 da 33 65 92 d7 c0 00                    3e..3e....
\end{lstlisting}

\subsection{Ricezione dati}
\label{sub:packetRicezione}

La serializzazione dei dati avviene attraverso il formato \verb+JSON+.
I dettagli dei pacchetti sono apprezzabili nell'appendice \ref{app:packetRicezione}.

\subsection{Chiusura della connessione}
\label{sub:packetChiusura}

Per entrambi i protocolli, la chiusura della connessione (senza interruzione forzata) viene semplicemente notificata al \textit{server}.

\begin{lstlisting}[language=bash, label={lst:packetChiusuraHtttp}, captionpos=b, caption={HTTP Chiusura della connessione}, basicstyle=\scriptsize\ttfamily]
0000   02 00 00 00 45 00 00 38 28 69 40 00 40 06 00 00  ....E..8(i@.@...
0010   7f 00 00 01 7f 00 00 01 0b b8 dc 7f 01 d3 99 54  ...............T
0020   8d de 93 c2 80 18 31 c1 fe 2c 00 00 01 01 08 0a  ......1..,......
0030   33 3d f1 42 33 3d f1 42 88 02 03 e9              3=.B3=.B....
\end{lstlisting}

\begin{lstlisting}[language=bash, label={lst:packetChiusuraMqtt}, captionpos=b, caption={MQTT Chiusura della connessione}, basicstyle=\scriptsize\ttfamily]
0000   02 00 00 00 45 00 00 36 8f 40 40 00 40 06 00 00  ....E..6.@@.@...
0010   7f 00 00 01 7f 00 00 01 de 5b 07 5b 56 1f e3 9e  .........[.[V...
0020   b5 ad d2 59 80 18 31 d0 fe 2a 00 00 01 01 08 0a  ...Y..1..*......
0030   33 65 f9 2a 33 65 eb 75 e0 00                    3e.*3e.u..
\end{lstlisting}
